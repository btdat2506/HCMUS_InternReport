% !TEX encoding = UTF-8 Unicode
% Ensure this file is saved with UTF-8 encoding

% \chapter{GIỚI THIỆU} % Keep this if it's the actual first chapter
% \label{Chapter1} % Keep label consistent if referenced elsewhere

% If this CESLAB intro is *part* of Chapter 1, use \section instead of \chapter
% Example:
\chapter{GIỚI THIỆU VỀ ĐƠN VỊ THỰC TẬP}
\label{Chapter1} % Label for the whole introduction chapter

\section{Giới thiệu chung}
\label{subsec:ceslab_general}

\begin{description}
    \item[Tên đơn vị:] Phòng thí nghiệm Máy Tính và Hệ Thống Nhúng (\acrshort{ceslab}).
    \item[Địa chỉ:] Phòng E103A, Khu nhà E, Trường Đại học Khoa học Tự nhiên, \acrshort{vnu-hcm}, 227 Nguyễn Văn Cừ, Phường 4, Quận 5, TP. Hồ Chí Minh.
    \item[Trực thuộc:] Bộ môn Máy tính - Hệ thống Nhúng, Khoa Điện tử - Viễn thông (\acrshort{fetel}), Trường Đại học Khoa học Tự nhiên (\acrshort{hcmus}), \acrshort{vnu-hcm}.
    \item[Website (tham khảo):] \url{https://www.ceslab.id.vn/}
    \item[Email (tham khảo):] \texttt{ceslab.fetel@gmail.com}
\end{description}

\acrshort{ceslab} được thành lập với mục tiêu trở thành trung tâm nghiên cứu và đào tạo chuyên sâu, uy tín trong lĩnh vực hệ thống nhúng (embedded systems), thiết kế vi mạch (\acrshort{ic} Design), và các công nghệ liên quan. Phòng thí nghiệm tạo môi trường học tập, thực hành và nghiên cứu hiện đại cho sinh viên, học viên cao học và nghiên cứu sinh, đồng thời thúc đẩy hợp tác với các doanh nghiệp và tổ chức trong và ngoài nước.

%\textit{(Lưu ý: Thông tin tìm kiếm chưa cung cấp chi tiết về năm thành lập, tên trưởng phòng thí nghiệm hiện tại, hoặc các dự án/thành tựu cụ thể. Cần bổ sung thêm thông tin này nếu có để phần giới thiệu đầy đủ hơn).}

\section{Lĩnh vực hoạt động và Nghiên cứu}
\label{subsec:ceslab_activities}

\acrshort{ceslab} tập trung vào các hướng nghiên cứu, phát triển và đào tạo chính trong các lĩnh vực công nghệ cao, bao gồm:

\begin{enumerate}
    \item \textbf{Thiết kế Hệ thống nhúng (\textit{Embedded Systems Design}):} Nghiên cứu, thiết kế và phát triển các hệ thống nhúng hoàn chỉnh, từ phần cứng đến phần mềm, ứng dụng trong nhiều lĩnh vực như công nghiệp, y tế, tiêu dùng.
    \item \textbf{Thiết kế Vi mạch và Hệ thống trên chip (\textit{\acrshort{ic} Design \& System-on-Chip - \acrshort{soc}}):} Tập trung vào thiết kế vi mạch số (digital IC design), vi mạch tương tự (analog IC design), tín hiệu hỗn hợp (mixed-signal) và tích hợp các hệ thống phức tạp trên một vi mạch duy nhất (\acrshort{soc}) sử dụng các công nghệ \acrshort{fpga} và \acrshort{asic}.
    \item \textbf{Lập trình Hệ thống nhúng và Hệ điều hành Thời gian thực (\textit{Embedded Software \& \acrshort{rtos}}):} Phát triển phần mềm điều khiển cấp thấp (low-level control software), trình điều khiển thiết bị (device drivers), và ứng dụng trên các nền tảng nhúng, bao gồm cả việc sử dụng hệ điều hành thời gian thực (\acrshort{rtos}).
    \item \textbf{Internet of Things (\acrshort{iot}):} Nghiên cứu và triển khai các giải pháp \acrshort{iot}, bao gồm thiết kế thiết bị biên (edge devices), mạng cảm biến không dây (wireless sensor networks), và các nền tảng thu thập, xử lý dữ liệu.
    \item \textbf{Xử lý Tín hiệu Số trên Phần cứng (\textit{Hardware-based Digital Signal Processing - \acrshort{dsp}}):} Tối ưu và triển khai các thuật toán xử lý tín hiệu số phức tạp trên các nền tảng phần cứng như \acrshort{fpga} và bộ xử lý \acrshort{dsp}.
    \item \textbf{Trí tuệ Nhân tạo cho Hệ thống nhúng (\textit{\acrshort{ai} for Embedded Systems / Edge AI}):} Nghiên cứu và ứng dụng các mô hình học máy (machine learning), học sâu (deep learning) trên các thiết bị nhúng có tài nguyên hạn chế, phục vụ các bài toán như nhận dạng hình ảnh (image recognition).
\end{enumerate}

Trong quá trình thực tập tại \acrshort{ceslab}, sinh viên được tiếp cận với các công cụ thiết kế, mô phỏng và kiểm thử tiên tiến, làm việc trên các bo mạch phát triển \acrshort{fpga} hiện đại và được hướng dẫn bởi các giảng viên, nghiên cứu viên có kinh nghiệm. Đây là môi trường lý tưởng để sinh viên trau dồi kiến thức chuyên môn, rèn luyện kỹ năng thực hành và định hướng cho sự nghiệp tương lai trong ngành Điện tử - Viễn thông.


Danh sách các nghiên cứu viên chính tại \acrshort{ceslab}:

\begin{itemize}
    \item Trần Tuấn Kiệt, MSc. (\texttt{trtkiet@hcmus.edu.vn})
    \item Đặng Tấn Phát, MSc. (\texttt{dtphat@hcmus.edu.vn})
    \item Nguyễn Như Hoàng, MSc. (\texttt{nnhoang@hcmus.edu.vn})
    \item Huỳnh Thị Minh Tuyến, M2. (\texttt{htmtuyen@hcmus.edu.vn})
    \item Hồ Thanh Bảo, BSc. (\texttt{htbao@hcmus.edu.vn})
\end{itemize}
