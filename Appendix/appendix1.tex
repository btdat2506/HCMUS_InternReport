\appendix

\chapter{MÃ NGUỒN CỦA THIẾT KẾ}
\label{Appendix1}

\section{Mã Verilog Top-Level cho DMANiosV}
\label{app:verilog_top} % Changed label for clarity
\begin{lstlisting}[language=Verilog, caption={DMANiosV.v - Top Level Module}, label=lst:verilog_top]
module DMANiosV(input CLOCK_50, input [0:0] KEY);
  system Nios_system (
    .clk_clk      (CLOCK_50),
    .reset_reset_n(KEY[0])
  );
endmodule
\end{lstlisting}

\section{Mã Verilog DMA Controller}
\label{app:verilog_dmac}
\begin{lstlisting}[language=Verilog, caption={DMAController.v - DMA Controller Top Module}, label=lst:verilog_dmacontroller]
module DMAController (
    input          iClk,
    input          iReset_n,
    // Avalon Slave Interface
    input          iChipselect,
    input          iRead,
    input          iWrite,
    input  [2:0]   iAddress,
    input  [31:0]  iWritedata,
    output [31:0]  oReaddata,
    // Avalon Read Master Interface
    output         oRM_read,
    output [31:0]  oRM_readaddress,
    input          iRM_readdatavalid,
    input          iRM_waitrequest,
    input  [31:0]  iRM_readdata,
    // Avalon Write Master Interface
    output         oWM_write,
    output [31:0]  oWM_writeaddress,
    output [31:0]  oWM_writedata,
    output [3:0]   oWM_byteenable, // <<< ADDED PORT
    input          iWM_waitrequest
);
    // Internal Signals
    wire           Start;
    wire [31:0]    Length;
    wire [31:0]    RM_startaddress;
    wire [31:0]    WM_startaddress;
    wire           WM_done;
    // FIFO Signals
    wire           FF_almostfull;
    wire           FF_writerequest;
    wire [31:0]    FF_data;
    wire           FF_empty;
    wire [31:0]    FF_q;
    wire           FF_readrequest;

    // Instantiate CONTROL_SLAVE
    CONTROL_SLAVE u_CONTROL_SLAVE (
        .iClk(iClk),
        .iReset_n(iReset_n),
        .iChipselect(iChipselect),
        .iRead(iRead),
        .iWrite(iWrite),
        .iAddress(iAddress),
        .iWritedata(iWritedata),
        .oReaddata(oReaddata),
        .RM_startaddress(RM_startaddress),
        .WM_startaddress(WM_startaddress),
        .Length(Length),
        .Start(Start),
        .WM_done(WM_done)
    );

    // Instantiate READ_MASTER (Use revised version)
    READ_MASTER u_READ_MASTER (
        .iClk(iClk),
        .iReset_n(iReset_n),
        .Start(Start),
        .Length(Length),
        .RM_startaddress(RM_startaddress),
        .FF_almostfull(FF_almostfull),
        .FF_writerequest(FF_writerequest),
        .FF_data(FF_data),
        .oRM_read(oRM_read),
        .oRM_readaddress(oRM_readaddress),
        .iRM_readdatavalid(iRM_readdatavalid),
        .iRM_waitrequest(iRM_waitrequest),
        .iRM_readdata(iRM_readdata)
    );

    // Instantiate WRITE_MASTER (Use revised version with byteenable)
    WRITE_MASTER u_WRITE_MASTER (
        .iClk(iClk),
        .iReset_n(iReset_n),
        .Start(Start),
        .Length(Length),
        .WM_startaddress(WM_startaddress),
        .FF_empty(FF_empty),
        .FF_readrequest(FF_readrequest),
        .FF_q(FF_q),
        .oWM_write(oWM_write),
        .oWM_writeaddress(oWM_writeaddress),
        .oWM_writedata(oWM_writedata),
        .oWM_byteenable(oWM_byteenable), // <<< CONNECTED PORT
        .iWM_waitrequest(iWM_waitrequest),
        .WM_done(WM_done)
    );

    // Instantiate FIFO
    wire FF_full;
    FIFO_IP	FIFO_IP_inst (
        .aclr (~iReset_n),
        .clock ( iClk ),
        .data ( FF_data ),
        .rdreq ( FF_readrequest ),
        .wrreq ( FF_writerequest ),
        .almost_full ( FF_almostfull ),
        .empty ( FF_empty ),
        .full ( FF_full ),
        .q ( FF_q )
	);

endmodule
\end{lstlisting}

\section{Mã Verilog Control Slave}
\label{app:verilog_control_slave}
\begin{lstlisting}[language=Verilog, caption={CONTROL\_SLAVE.v - Avalon Slave for Control/Status}, label=lst:verilog_controlslave]
module CONTROL_SLAVE (
    input          iClk,
    input          iReset_n,
    // Avalon Slave Interface
    input          iChipselect,
    input          iRead,
    input          iWrite,
    input  [2:0]   iAddress,
    input  [31:0]  iWritedata,
    output [31:0]  oReaddata,
    // DMA Configuration/Control Outputs
    output [31:0]  RM_startaddress,
    output [31:0]  WM_startaddress,
    output [31:0]  Length,
    output         Start,
    // DMA Status Input
    input          WM_done
);
    // Parameter Definitions for Addresses
    localparam ADDR_READADDRESS  = 3'd0;
    localparam ADDR_WRITEADDRESS = 3'd1;
    localparam ADDR_LENGTH       = 3'd2;
    // Address 3'd3 is unused
    localparam ADDR_CONTROL      = 3'd4;
    localparam ADDR_STATUS       = 3'd5;
    // Addresses 3'd6, 3'd7 are unused

    // Register Definitions
    reg [31:0] readaddress_reg;
    reg [31:0] writeaddress_reg;
    reg [31:0] length_reg;
    reg        control_go;      // Internal GO signal, latched
    reg        status_busy;
    reg        status_done;

    // Output Assignments
    assign RM_startaddress = readaddress_reg;
    assign WM_startaddress = writeaddress_reg;
    assign Length          = length_reg;
    assign Start           = control_go; // Start signal follows the latched GO bit

    // Read Data Output Logic (Combinatorial)
    reg [31:0] readdata_reg;
    assign oReaddata = readdata_reg;

    // Combinatorial block for read data multiplexing
    always @(*) begin
        // Default read data to 0
        readdata_reg = 32'd0;
        if (iChipselect && iRead) begin
            case (iAddress)
                ADDR_READADDRESS:  readdata_reg = readaddress_reg;
                ADDR_WRITEADDRESS: readdata_reg = writeaddress_reg;
                ADDR_LENGTH:       readdata_reg = length_reg;
                ADDR_CONTROL:      readdata_reg = {31'd0, control_go};      // Return current GO bit status
                ADDR_STATUS:       readdata_reg = {30'd0, status_busy, status_done}; // Return Busy and Done status
                default:           readdata_reg = 32'd0; // Reads from unused addresses return 0
            endcase
        end
    end

    // Register Write and State Logic (Sequential)
    always @(posedge iClk or negedge iReset_n) begin
        if (!iReset_n) begin
            readaddress_reg  <= 32'd0;
            writeaddress_reg <= 32'd0;
            length_reg       <= 32'd0;
            control_go       <= 1'b0;
            status_busy      <= 1'b0;
            status_done      <= 1'b0;
        end else begin
            // Handle register writes from CPU
            if (iChipselect && iWrite) begin
                case (iAddress)
                    ADDR_READADDRESS:  readaddress_reg  <= iWritedata;
                    ADDR_WRITEADDRESS: writeaddress_reg <= iWritedata;
                    ADDR_LENGTH:       length_reg       <= iWritedata;
                    ADDR_CONTROL:      control_go       <= iWritedata[0]; // Latch the GO bit from CPU
                    ADDR_STATUS: begin
                        // Bit 0: Write 1 to clear Done status
                        if (iWritedata[0]) begin
                            status_done <= 1'b0;
                        end
                        // Other bits are read-only or unused
                    end
                    default: ; // No action for unused addresses
                endcase
            end

            if (control_go && !status_busy && !status_done) begin
                status_busy <= 1'b1;
                status_done <= 1'b0;
            end

            // If the Write Master signals completion THIS cycle
            if (WM_done) begin
                status_busy <= 1'b0;
                status_done <= 1'b1;
                control_go  <= 1'b0; // <<< Clear GO bit automatically on completion
            end

            // If CPU clears GO bit while busy, we might need an abort mechanism (currently ignored)
        end
    end

endmodule
\end{lstlisting}

\section{Mã Verilog Read Master}
\label{app:verilog_read_master}
\begin{lstlisting}[language=Verilog, caption={READ\_MASTER.v - Avalon Read Master Module}, label=lst:verilog_readmaster]
module READ_MASTER (
    input          iClk,
    input          iReset_n,
    input          Start,
    input  [31:0]  Length,
    input  [31:0]  RM_startaddress,
    input          FF_almostfull,
    output reg     FF_writerequest,
    output reg [31:0] FF_data,
    output reg     oRM_read,
    output reg [31:0] oRM_readaddress,
    input          iRM_readdatavalid,
    input          iRM_waitrequest,
    input  [31:0]  iRM_readdata
);
    // State Machine States
    parameter IDLE = 3'b000, CHECK_FIFO = 3'b01, REQUEST = 3'b010, WAIT_DATA = 3'b011, WRITE_FIFO = 3'b100, WAIT_FIFO = 3'b101;
    reg [2:0] current_state, next_state;

    // Internal Registers
    reg [31:0] bytes_remaining;
    reg [31:0] total_bytes;

    // Initialization
    always @(posedge iClk or negedge iReset_n) begin
        if (!iReset_n) begin
            current_state    <= IDLE;
            oRM_read         <= 1'b0;
            oRM_readaddress  <= 32'd0;
            FF_writerequest  <= 1'b0;
            FF_data          <= 32'd0;
            bytes_remaining  <= 32'd0;
            total_bytes      <= 32'd0;
        end else begin
            current_state <= next_state;
            FF_writerequest <= 1'b0;            

            case (current_state)
                IDLE: begin
                    oRM_read        <= 1'b0;
                    FF_writerequest <= 1'b0;
                    if (Start && oRM_readaddress < RM_startaddress + Length) begin
                        bytes_remaining <= Length;
                        oRM_readaddress <= RM_startaddress;
                        total_bytes     <= Length;
                    end
                end
                CHECK_FIFO: begin
                    if (!FF_almostfull) begin
                        oRM_read <= 1'b1;
                    end else begin
                        oRM_read <= 1'b0;
                    end
                end
                REQUEST, WAIT_DATA: begin
                    
                    //oRM_read <= 1'b0;
                end
                WRITE_FIFO: begin
                    if (iRM_readdatavalid) begin
                        //oRM_read        <= 1'b0;
                        FF_writerequest <= 1'b1;
                        FF_data         <= iRM_readdata;
                    end else begin
                        FF_writerequest <= 1'b0;
                    end
                end
                WAIT_FIFO: begin
                    oRM_readaddress <= oRM_readaddress + 4;
                    bytes_remaining <= bytes_remaining - 4;
                end
            endcase
        end
    end

    // Next State Logic
    always @(*) begin
        case (current_state)
            IDLE: begin
                if (Start && oRM_readaddress < RM_startaddress + Length && !FF_almostfull)
                    next_state = CHECK_FIFO;
                else
                    next_state = IDLE;
            end
            CHECK_FIFO: begin
                if (!FF_almostfull)
                    next_state = REQUEST;
                else
                    next_state = CHECK_FIFO;
            end
            REQUEST: begin
                if (iRM_waitrequest)
                    next_state = REQUEST;
                else
                    next_state = WAIT_DATA;
            end
            WAIT_DATA: begin
                if (iRM_readdatavalid)
                    next_state = WRITE_FIFO;
                else
                    next_state = WAIT_DATA;
            end
            WRITE_FIFO: begin
                if (iRM_readdatavalid) begin
                    next_state = WAIT_FIFO;
                end else begin
                    next_state = WRITE_FIFO;
                end
                /* if (bytes_remaining == 32'd0 || oRM_readaddress == RM_startaddress + Length || FF_almostfull)
                    next_state = IDLE;
                else
                    next_state = REQUEST; */
                //next_state = WAIT_FIFO;
            end
            WAIT_FIFO: begin
                if (bytes_remaining == 32'd0 || oRM_readaddress == RM_startaddress + Length || FF_almostfull)
                    next_state = IDLE;
                else
                    next_state = REQUEST;
            end
            default: next_state = IDLE;
        endcase
    end
endmodule
\end{lstlisting}

\section{Mã Verilog Write Master}
\label{app:verilog_write_master}
\begin{lstlisting}[language=Verilog, caption={WRITE\_MASTER.v - Avalon Write Master Module}, label=lst:verilog_writemaster]
module WRITE_MASTER (
    input          iClk,
    input          iReset_n,
    // Control Inputs
    input          Start,           // From CONTROL_SLAVE
    input  [31:0]  Length,          // Total bytes to write
    input  [31:0]  WM_startaddress, // Start address for writing
    // FIFO Interface
    input          FF_empty,        // FIFO empty signal
    output reg     FF_readrequest,  // Request to read from FIFO
    input  [31:0]  FF_q,            // Data read from FIFO
    // Avalon Write Master Interface
    output reg     oWM_write,       // Avalon write signal
    output reg [31:0] oWM_writeaddress, // Avalon write address
    output reg [31:0] oWM_writedata,   // Avalon write data
    output reg [3:0]  oWM_byteenable, // Avalon byte enable
    input          iWM_waitrequest, // Avalon wait request
    // Status Output
    output reg     WM_done          // DMA Write Master Done
);
    // State Machine States
    parameter IDLE           = 3'b000;
    parameter CHECK_FIFO     = 3'b001; // Check if data is available in FIFO
    parameter READ_FIFO      = 3'b010; // Request data from FIFO
    parameter WAIT_FIFO_DATA = 3'b011; // Wait for FIFO data (usually available next cycle)
    parameter START_WRITE    = 3'b100; // Assert write, address, data, byteenable
    parameter WAIT_WRITE_ACK = 3'b101; // Wait for waitrequest=0 first time
    parameter UPDATE_CNT     = 3'b111; // Update counters, check completion

    reg [2:0] current_state, next_state;

    // Internal Registers
    reg [31:0] bytes_remaining;
    reg [31:0] current_address;
    reg [31:0] data_to_write; // Register to hold data from FIFO

    // Initialization and State Register
    always @(posedge iClk or negedge iReset_n) begin
        if (!iReset_n) begin
            current_state   <= IDLE;
            bytes_remaining <= 32'd0;
            current_address <= 32'd0;
            WM_done         <= 1'b0;
            // *** ADD RESET FOR data_to_write ***
            data_to_write   <= 32'd0;
            // *** Outputs are reset below ***
        end else begin
            current_state <= next_state;
            // Update counters in UPDATE_CNT state
            if (current_state == UPDATE_CNT) begin
                current_address <= current_address + 4;
                bytes_remaining <= bytes_remaining - 4;
            end
            // Latch configuration in IDLE state when starting
            if (next_state == CHECK_FIFO && current_state == IDLE) begin
                bytes_remaining <= Length;
                current_address <= WM_startaddress;
                WM_done         <= 1'b0; // Clear done when starting a new transfer
            end
            // Set WM_done when transitioning to IDLE from UPDATE_CNT (last word processed)
            if (current_state == UPDATE_CNT && bytes_remaining <= 4) begin
                 WM_done <= 1'b1;
            end
            // Latch data from FIFO in WAIT_FIFO_DATA state (uses state *before* edge)
            // Moved this inside the 'else' block for clarity, happens concurrently
            if (current_state == READ_FIFO) begin
                 data_to_write <= FF_q;
            end
        end
    end

    // Output Logic (Registered Outputs)
    always @(posedge iClk or negedge iReset_n) begin
        if (!iReset_n) begin
            FF_readrequest   <= 1'b0;
            oWM_write        <= 1'b0;
            oWM_writeaddress <= 32'd0;
            oWM_writedata    <= 32'd0;
            oWM_byteenable   <= 4'b0000;
        end else begin
            // Default assignments for next cycle (registered outputs)
            FF_readrequest <= 1'b0;
            oWM_write      <= 1'b0;
            oWM_byteenable <= 4'b0000;
            // oWM_writeaddress and oWM_writedata hold previous value unless changed

            // Drive outputs based on state (using state *before* edge)
            case (current_state)
                READ_FIFO: begin // State 2
                    FF_readrequest <= 1'b1; // Request data for next cycle
                end
                START_WRITE: begin // State 4
                    oWM_write        <= 1'b1;
                    oWM_writeaddress <= current_address;
                    oWM_writedata    <= FF_q; // Use value latched on *previous* edge
                    oWM_byteenable   <= 4'b1111;
                end
                WAIT_WRITE_ACK: begin // State 5
                    oWM_write        <= 1'b1;          // Keep write asserted
                    oWM_writeaddress <= current_address; // Keep address stable
                    oWM_writedata    <= FF_q;   // Keep data stable
                    oWM_byteenable   <= 4'b1111;      // Keep byteenable asserted
                end
                // In other states, outputs driven low/inactive by defaults above
                default: begin
                end
            endcase
        end
    end

    // Next State Logic (Combinatorial)
    always @(*) begin
        next_state = current_state; // Default to stay in current state
        if (!Start) begin
            next_state = IDLE; // Reset to IDLE if Start is deasserted
        end else
        case (current_state)
            IDLE: begin // State 0
                if (Start && Length > 0) begin
                    next_state = CHECK_FIFO; // -> 1
                end else begin
                    next_state = IDLE; // -> 0
                end
            end
            CHECK_FIFO: begin // State 1
                if (!FF_empty) begin
                    next_state = READ_FIFO; // -> 2
                end else begin
                    if (bytes_remaining > 0 && Start) begin
                       next_state = CHECK_FIFO; // -> 1
                    end else begin
                       next_state = IDLE; // -> 0
                    end
                end
            end
            READ_FIFO: begin // State 2
                next_state = WAIT_FIFO_DATA; // -> 3
            end
            WAIT_FIFO_DATA: begin // State 3
                 next_state = START_WRITE; // -> 4
            end
            START_WRITE: begin // State 4
                // Immediately move to wait if write is asserted
                next_state = WAIT_WRITE_ACK; // -> 5
            end
            WAIT_WRITE_ACK: begin // State 5
                if (!iWM_waitrequest) begin
                    next_state = UPDATE_CNT; // -> 6 (Write accepted)
                end else begin
                    next_state = WAIT_WRITE_ACK; // -> 5 (Keep waiting)
                end
            end
            UPDATE_CNT: begin // State 6
                if (bytes_remaining <= 4) begin // Check before decrement
                    next_state = IDLE; // -> 0 (Transfer complete)
                end else begin
                    next_state = CHECK_FIFO; // -> 1 (More data to write)
                end
            end
            default: next_state = IDLE;
        endcase

    end

endmodule
\end{lstlisting}

\section{Mã nguồn C kiểm thử DMA}
\label{app:c_code}
\begin{lstlisting}[language=C, caption={hello\_world.c - Nios V DMA Test Application}, label=lst:c_helloworld]
#include <stdio.h>
#include <stdint.h>
#include <stdbool.h>
#include "system.h"        // System-specific definitions
#include "io.h"            // I/O functions (if needed)
#include "sys/alt_cache.h" // <<< Cache management functions
#include "alt_types.h"     // <<< For alt_u32

#define DMA_REG_READADDRESS 0  // Offset 0 (Byte Address 0x00)
#define DMA_REG_WRITEADDRESS 1 // Offset 1 (Byte Address 0x04)
#define DMA_REG_LENGTH 2       // Offset 2 (Byte Address 0x08)
// Offset 3 is unused
#define DMA_REG_CONTROL 4 // Offset 4 (Byte Address 0x10)
#define DMA_REG_STATUS 5  // Offset 5 (Byte Address 0x14)

// Control Register Bits
#define DMA_CONTROL_GO (1 << 0)

// Status Register Bits
#define DMA_STATUS_DONE (1 << 0)
#define DMA_STATUS_BUSY (1 << 1)

// Reset Timeout (number of status polls)
#define DMA_RESET_TIMEOUT 10000

// Buffer size
#define BUFFER_SIZE 32

#define arg_simulation 1 // Set to 0 for simulation, 1 for hardware

// Source data buffer (likely placed in onchip_memory2_0 by linker)
// Initialized with values 2 to 33
alt_u32 pdata0[BUFFER_SIZE] = {
    2, 3, 4, 5, 6, 7, 8, 9,
    10, 11, 12, 13, 14, 15, 16, 17,
    18, 19, 20, 21, 22, 23, 24, 25,
    26, 27, 28, 29, 30, 31, 32, 33};

// Destination data buffer pointer - explicitly placed in onchip_memory2_1
// Adding an offset within the memory for clarity/safety
volatile alt_u32 *pdata1 = (alt_u32*) (ONCHIP_MEMORY2_1_BASE + 0x200); // Offset 0x100 within mem 1

// Write to DMA register (using word offset for IOWR)
static inline void dma_write_reg(alt_u32 reg_offset, alt_u32 value)
{
    IOWR(DMA_CONTROLLER_0_BASE, reg_offset, value);
}

// Read from DMA register (using word offset for IORD)
static inline alt_u32 dma_read_reg(alt_u32 reg_offset)
{
    return IORD(DMA_CONTROLLER_0_BASE, reg_offset);
}

// --- NEW FUNCTION: Reset DMA Controller ---
// Attempts to reset the DMA state via software writes
bool reset_dma()
{
    printf("Resetting DMA Controller...\n");

    // 1. Clear the GO bit in the Control Register to stop new transfers
    printf("  Clearing GO bit (Reg %d)...\n", DMA_REG_CONTROL);
    dma_write_reg(DMA_REG_CONTROL, 0);

    // 2. Clear configuration registers
    printf("  Clearing Address/Length Regs (%d, %d, %d)...\n",
           DMA_REG_READADDRESS, DMA_REG_WRITEADDRESS, DMA_REG_LENGTH);
    dma_write_reg(DMA_REG_READADDRESS, 0);
    dma_write_reg(DMA_REG_WRITEADDRESS, 0);
    dma_write_reg(DMA_REG_LENGTH, 0);

    // 3. Clear the DONE status bit (Write-1-to-clear)
    // Check if DONE is already set before clearing
    if (dma_read_reg(DMA_REG_STATUS) & DMA_STATUS_DONE)
    {
        printf("  Clearing DONE status bit (Reg %d)...\n", DMA_REG_STATUS);
        dma_write_reg(DMA_REG_STATUS, DMA_STATUS_DONE); // Write 1 to DONE bit
    }

    // 4. Wait for the BUSY flag to clear (optional but recommended)
    printf("  Waiting for BUSY bit to clear (Reg %d)...\n", DMA_REG_STATUS);
    int timeout = DMA_RESET_TIMEOUT;
    while ((dma_read_reg(DMA_REG_STATUS) & DMA_STATUS_BUSY) && (timeout > 0))
    {
        timeout--;
        // Optional delay
        // for(volatile int d=0; d<10; d++);
    }

    if (timeout == 0)
    {
        printf("  Error: DMA reset timeout - BUSY bit did not clear!\n");
        return false;
    }

    printf("DMA Reset complete. Final Status: 0x%lx\n", dma_read_reg(DMA_REG_STATUS));
    return true;
}
// --- End of reset_dma function ---

// Start DMA transfer
void start_dma_transfer(alt_u32 src_addr, alt_u32 dst_addr, alt_u32 length_bytes)
{
    printf("Configuring DMA Regs:\n");
    printf("  Reg %d (Read Addr) : 0x%08lx\n", DMA_REG_READADDRESS, src_addr);
    printf("  Reg %d (Write Addr): 0x%08lx\n", DMA_REG_WRITEADDRESS, dst_addr);
    printf("  Reg %d (Length)    : %lu bytes\n", DMA_REG_LENGTH, length_bytes);
    printf("  Reg %d (Control)   : 0x%08X\n", DMA_REG_CONTROL, (alt_u32)DMA_CONTROL_GO);

    // Write the source and destination addresses
    dma_write_reg(DMA_REG_READADDRESS, src_addr);
    dma_write_reg(DMA_REG_WRITEADDRESS, dst_addr);

    // Write the length (number of bytes to transfer)
    dma_write_reg(DMA_REG_LENGTH, length_bytes);
    dma_write_reg(DMA_REG_CONTROL, DMA_CONTROL_GO);
    printf("DMA GO bit set.\n");
}

// Wait for DMA to complete
bool wait_for_dma_done()
{
    alt_u32 status;
    printf("Polling DMA Status (Reg %d)...\n", DMA_REG_STATUS);
    // Add a timeout to prevent infinite loops
    int timeout = 1000000; // Adjust as needed
    while (timeout > 0)
    {
        status = dma_read_reg(DMA_REG_STATUS);
        // printf("Status: 0x%08lx\n", status); // Debug print
        if (status & DMA_STATUS_DONE)
        {
            printf("DMA DONE bit detected.\n");
            // Optional: Clear the DONE bit after detecting it by writing 1 to it
            // dma_write_reg(DMA_REG_STATUS, DMA_STATUS_DONE);
            return true;
        }
        timeout--;
    }
    printf("Error: Timeout waiting for DMA DONE bit. Final Status: 0x%lx\n", status);
    return false;
}

// Verify DMA transfer
bool verify_dma_transfer(alt_u32 *src, alt_u32 *dst, alt_u32 num_words)
{
    bool success = true;
    printf("Verifying %lu words...\n", num_words);
    for (int i = 0; i < num_words; i++)
    {
        // Use volatile pointers for reading destination to bypass cache optimization if needed,
        // although cache flush is the proper method.
        volatile alt_u32 read_data = dst[i];
        if (read_data != src[i])
        {
            printf("Mismatch at index %d: expected 0x%08lX, got 0x%08lX\n",
                   i, (alt_u32)src[i], (alt_u32)read_data);
            success = false;
            // return false; // Optionally exit on first mismatch
        }
        else
        	printf("index %d = %d\n", i, *(dst + i));
    }
    return success;
}

int main()
{
    alt_u32 src_address = (alt_u32)pdata0;
    alt_u32 dst_address = (alt_u32)pdata1;

    // Transfer size in bytes
    alt_u32 length_bytes = BUFFER_SIZE * sizeof(alt_u32);

    if (arg_simulation)
    {
        printf("=====================================================\n");
        printf("= 21207001 - Bui Thanh Dat - HCMUS - FETEL - CESLAB =\n");
        printf("= Nios V DMA Example =\n");
        printf("=====================================================\n");
        // --- Reset DMA at the start ---
        if (!reset_dma())
        {
            printf("DMA failed to reset. Halting.\n");
            return -1;
        }
        // --- DMA should be idle now ---

        // Ensure source data is flushed from cache *before* DMA reads it
        alt_dcache_flush((void *)pdata0, sizeof(pdata0));

        printf("\nSystem Base Addresses:\n");
        printf("  ONCHIP_MEMORY2_0_BASE: 0x%08lx\n", (alt_u32)ONCHIP_MEMORY2_0_BASE);
        printf("  ONCHIP_MEMORY2_1_BASE: 0x%08lx\n", (alt_u32)ONCHIP_MEMORY2_1_BASE);
        printf("  DMA_CONTROLLER_0_BASE:   0x%08lx\n", (alt_u32)DMA_CONTROLLER_0_BASE);
        printf("Actual pdata0 address: %p\n", pdata0); // Verify source location
        printf("Actual pdata1 address: %p\n", pdata1); // Verify destination location

        // Check if DMA is busy before starting a new transfer (shouldn't be after reset)
        if (dma_read_reg(DMA_REG_STATUS) & DMA_STATUS_BUSY)
        {
            printf("Error: DMA is busy even after reset! Status: 0x%lx\n", dma_read_reg(DMA_REG_STATUS));
            return -1;
        }

        // Start DMA transfer
        printf("\nStarting DMA transfer...\n");
        printf("Source Address: 0x%08lx (pdata0)\n", src_address);
        printf("Destination Address: 0x%08lx (pdata1)\n", dst_address);
        printf("Length: %lu bytes (%d words)\n", length_bytes, BUFFER_SIZE);

        // Print destination buffer content *before* clearing and transfer
        // Note: This read might be from cache if previously accessed
        printf("\nDestination Buffer Before Clearing (potentially cached):\n");
        for (int i = 0; i < BUFFER_SIZE; i++)
        {
            printf("  index %2d: 0x%08lx\n", i, (alt_u32)pdata1[i]);
            if ((i + 1) % 8 == 0)
                printf("\n"); // Formatting
        }

        // Reset the destination buffer using CPU writes
        printf("\nResetting destination buffer via CPU...\n");
        for (int i = 0; i < BUFFER_SIZE; i++)
        {
            pdata1[i] = 0;
        }
        // Flush the cache *after* CPU writes zeros to ensure zeros are in memory
        alt_dcache_flush((void *)dst_address, length_bytes);
        printf("Destination buffer cleared and cache flushed.\n");

        // Optional: Read back immediately after flush to confirm zeros
        printf("\nDestination Buffer After Reset & Flush (read back):\n");
        for (int i = 0; i < BUFFER_SIZE; i++)
        {
            printf("  index %2d: 0x%08lx\n", i, (alt_u32)pdata1[i]);
            if ((i + 1) % 8 == 0)
                printf("\n"); // Formatting
        }
    }

    // Start the actual DMA
    start_dma_transfer(src_address, dst_address, length_bytes);

    // Wait for DMA to complete
    if (!wait_for_dma_done())
    {
        printf("DMA transfer failed or timed out!\n");
        // Optional: Attempt another reset here?
        // reset_dma();
        return -1;
    }

    // !!! Crucial: Flush cache for destination region BEFORE verification !!!
    printf("\nFlushing D-Cache for destination buffer...\n");
    alt_dcache_flush((void *)dst_address, length_bytes);
    printf("D-Cache flushed.\n");

    // Verify DMA transfer by reading memory (now hopefully coherent)
    printf("\nVerifying DMA transfer...\n");
    if (verify_dma_transfer(pdata0, pdata1, BUFFER_SIZE))
    {
        printf("DMA Transfer Verification Successful!\n");
    }
    else
    {
        printf("DMA Transfer Verification Failed!\n");
    }

    return 0;
}
\end{lstlisting}
