\chapter{KẾT QUẢ ĐẠT ĐƯỢC VÀ THÁI ĐỘ} % Suggested Chapter Title
\label{Chapter5}

\section{Về kiến thức} % Renumbered subsection
\label{sec:knowledge_gained}

Qua quá trình thực tập tại \acrshort{ceslab}, tôi đã đạt được những kiến thức chuyên môn sau:

\begin{itemize}
    \item \textbf{Kiến thức chuyên sâu về bộ xử lý Nios II\footnote{Mặc dù dự án chính tập trung vào Nios V, quá trình tìm hiểu và so sánh có thể bao gồm cả Nios II.}}: Hiểu rõ về kiến trúc, cách thức hoạt động và các thành phần chính của bộ xử lý \acrshort{nios2}. Nắm vững cách cấu hình và tùy chỉnh bộ xử lý cho các ứng dụng cụ thể. 
    \item \textbf{Hiểu biết về \acrshort{dma} và cơ chế truyền dữ liệu:} Hiểu rõ về cơ chế hoạt động của \acrshort{dma}, các loại truyền dữ liệu và các phương pháp tối ưu hóa hiệu suất truyền dữ liệu.
    \item \textbf{Kiến thức về thiết kế hệ thống nhúng:} Nắm vững các nguyên tắc thiết kế hệ thống nhúng, cách tổ chức và quản lý bộ nhớ, cũng như các kỹ thuật tối ưu hóa hiệu suất.
    \item \textbf{Hiểu biết về lập trình nhúng:} Phát triển kỹ năng lập trình cho hệ thống nhúng, bao gồm lập trình driver, lập trình giao tiếp với phần cứng và xử lý các vấn đề về đồng bộ hóa.
    \item \textbf{Kiến thức về các công cụ phát triển:} Làm quen với các công cụ phát triển phần cứng và phần mềm cho hệ thống nhúng, bao gồm Quartus Prime, ModelSim/Questa Sim, và các công cụ phát triển phần mềm cho \acrshort{niosv} (như Ashling RiscFree™ \acrshort{ide}). 
\end{itemize}

\section{Về kỹ năng}
\label{sec:skills_developed}

Ngoài những kiến thức chuyên môn, tôi cũng đã phát triển và cải thiện nhiều kỹ năng quan trọng:

\begin{itemize}
    \item \textbf{Kỹ năng phân tích và giải quyết vấn đề:} Học được cách phân tích các vấn đề phức tạp, xác định nguyên nhân gốc rễ và đề xuất các giải pháp hiệu quả.
    \item \textbf{Kỹ năng lập trình và debug:} Cải thiện khả năng viết mã sạch, dễ bảo trì và có khả năng mở rộng.
    \item \textbf{Kỹ năng thiết kế và tối ưu hóa hệ thống:} Học được cách thiết kế hệ thống đáp ứng các yêu cầu kỹ thuật và tối ưu hóa hiệu suất bằng cách điều chỉnh các tham số hệ thống (ví dụ: cấu hình \acrshort{ip} trong Platform Designer). 
    \item \textbf{Kỹ năng trình bày và báo cáo:} Cải thiện khả năng trình bày ý tưởng và kết quả một cách rõ ràng, ngắn gọn và thuyết phục thông qua việc chuẩn bị báo cáo này. 
\end{itemize}

\section{Về thái độ} 
\label{sec:attitude_developed}

Trong quá trình thực tập, tôi đã phát triển và thể hiện những thái độ tích cực sau:
\begin{itemize}
    \item \textbf{Tinh thần học hỏi và cầu tiến:} Luôn giữ tâm thế sẵn sàng học hỏi kiến thức mới, chủ động tìm hiểu thêm về các công nghệ và kỹ thuật mới liên quan đến \acrshort{fpga}, \acrshort{soc}, và RISC-V.
    \item \textbf{Tinh thần trách nhiệm:} Luôn có trách nhiệm với công việc được giao, hoàn thành đúng thời hạn và đạt chất lượng yêu cầu.
    \item \textbf{Sự kiên trì và kiên nhẫn:} Không nản lòng trước khó khăn, luôn kiên trì tìm hiểu và giải quyết các vấn đề phức tạp, đặc biệt là trong quá trình debug phần cứng và phần mềm.
    \item \textbf{Thái độ cầu thị và tiếp nhận phản hồi:} Luôn sẵn sàng tiếp nhận phản hồi từ người hướng dẫn, coi đó là cơ hội để học hỏi và cải thiện bản thân.
\end{itemize}

Những kết quả đạt được trong quá trình thực tập không chỉ giúp tôi hoàn thành tốt nhiệm vụ được giao mà còn là nền tảng vững chắc cho sự phát triển nghề nghiệp của tôi trong tương lai trong lĩnh vực thiết kế hệ thống nhúng và \acrshort{fpga}. 